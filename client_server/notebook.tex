% This is the bit of LaTeX style information that DataCell.cc needs in
% order to write notebooks out in standalone LaTeX form. It is very
% similar to ../frontend/common/preamble.tex; keep them in sync.

\documentclass[10pt]{article}
\usepackage[scale=.8]{geometry}
\usepackage{setspace}
\usepackage{fancyhdr}
\usepackage{listings}
\usepackage[fleqn]{amsmath}
\usepackage{color}
\usepackage{changepage}
\usepackage[colorlinks=true, urlcolor=black, plainpages=false, pdfpagelabels]{hyperref}
\usepackage{etoolbox}
\usepackage{amssymb}
\usepackage[parfill]{parskip}
\usepackage{graphicx}
%\usepackage{tableaux}
\def\specialcolon{\mathrel{\mathop{:}}\hspace{-.5em}}
\renewcommand{\bar}[1]{\overline{#1}}
\newcommand{\algorithm}[2]{{\tt\Large\detokenize{#1}}\\[1ex]
{\emph{#2}}\\[1ex]
}
\newcommand{\property}[2]{{\tt\Large\detokenize{#1}}\\[1ex]
{\emph{#2}}\\[1ex]
}
\newcommand{\algo}[1]{{\tt \detokenize{#1}}}
\newcommand{\prop}[1]{{\tt \detokenize{#1}}}
\newcommand{\author}[1]{{\bfseries #1}}
\newcommand{\email}[1]{, {\tt #1}}
%\makeatletter\def\old@comma{,}\catcode`\,=13 \def,{%
%\ifmmode\old@comma\discretionary{}{}{}\else\old@comma\fi}\makeatother

% Math expressions wrapped in \brwrap will get typeset with
% round brackets around them, which have the appropriate size.
% The expression itself can still be broken over multiple lines.

\newcommand\brwrap[3]{%
  \setbox0=\hbox{$#2$}
  \left#1\vbox to \the\ht0{\hbox to 0pt{}}\right.\kern-.2em
  \begingroup #2\endgroup\kern-.15em
  \left.\vbox to \the\ht0{\hbox to 0pt{}}\right#3
}

\renewcommand{\arraystretch}{1.2}
\tolerance=10000
\relpenalty=10
\binoppenalty=10
\hyphenpenalty=10
\raggedright

\lstnewenvironment{python}[1][]{\lstset{language=python,
   columns=fullflexible,
   xleftmargin=1em,
   belowskip=0pt,
   tabsize=3,
   commentstyle={}, % otherwise {#} cadabra arguments look ugly
   breaklines=true,   
   basicstyle=\small\ttfamily\color{blue},
   keywordstyle={}
}}{}
  

\everymath{\displaystyle}

% Page numbers
\pagestyle{fancy}
\fancyhf{} % clear all header and footer fields
\renewcommand{\headrulewidth}{0pt}
\renewcommand{\footrulewidth}{0pt}
\fancyfoot[LE,RO]{{\small\thepage}}
\fancyfoot[LO,RE]{{\tiny\href{http://cadabra.science}{{\tt http://cadabra.science}}}}

% \makeatletter\def\old@comma{,}\catcode`\,=13 \def,{%
% \ifmmode\old@comma\discretionary{}{}{}\else\old@comma\fi}\makeatother

% Ensure that maths broken over multiple lines has a bit of spacing
% between lines.
\lineskiplimit=0mm
\lineskip=1.5ex

% Typesetting Young tableaux, originally in a separate style
% file, now included to avoid path searching problems. 
% Some internals for the typesetting macros below; nothing
% user-servicable here; please read on.

\def\@tabforc#1#2#3{\expandafter\tabf@rc\expandafter#1{#2 \^}{#3}}
\def\tabf@@rc#1#2#3\tabf@@rc#4{\def#1{#2}#4\tabf@rc#1{#3}{#4}}
\long\def\ReturnAfterFi#1\fi{\fi#1}
    \def\tabf@rc#1#2#3{%
      \def\temp@ty{#2}%
      \ifx\@empty\temp@ty
      \else
        \ReturnAfterFi{%
          \tabf@@rc#1#2\tabf@@rc{#3}%
        }%
      \fi
    }%

% Sorry, some global registers for sizes and keeping track of
% measurements.
    
\newdimen\ytsize\ytsize=2mm
\newdimen\ytfsize\ytfsize=4mm
\newcount\repcnt
\newdimen\acchspace
\newdimen\accvspace
\newdimen\raiseh
\newdimen\maxw

\newcommand\phrule[1]{\hbox{\vbox to0pt{\hrule height .2pt width#1\vss}}}

% Typeset a Young tableau with filled boxes. Takes a single 
% argument which is a string of symbols for each row,
% separated by commas. Examples:
%
%   \ftableau{abc,de}
%   \ftableau{ab{d_2},f{g_3}}

\newcommand\ftableau[1]{%
\def\ctest{,}
\def\Ktest{\^}
\acchspace=0ex
\accvspace=0ex
\maxw=0ex
\vbox{\hbox{%
\@tabforc\thisel{#1}{%
 \ifx\thisel\Ktest{%
     \ifnum\maxw=0\maxw=\acchspace\fi%
     \raisebox{\accvspace}{\vbox to \ytfsize{\hbox to
		 0pt{\vrule height \ytfsize\hss}}}\kern\acchspace\kern-\maxw}
 \else\ifx\thisel\ctest
     \ifnum\maxw=0\maxw=\acchspace\fi%
     \raisebox{\accvspace}{\vbox to \ytfsize{\hbox to 0pt{\vrule height \ytfsize\hss}}}%
     \kern\acchspace\acchspace=0ex
	  \advance\accvspace by -\ytfsize
 \else
     \setbox3=\hbox{$\thisel$}%
	  \raiseh=\ytfsize%
	  \advance\raiseh by -1ex%
	  \divide\raiseh by 2%
     \advance\acchspace by-\ytfsize%
     \raisebox{\accvspace}{\vbox to \ytfsize{\hrule\hbox to%
        \ytfsize{\vrule height \ytfsize\hskip.5ex%
         \raisebox{\raiseh}{\tiny$\thisel$}\hss}\vss\phrule{\ytfsize}}}%
 \fi\fi}}}}

% Typeset a Young tableau with unlabelled boxes. Takes a single 
% argument which is a string of numbers, one for the length of
% each row of the tableau. Example:
%
%   \tableau{{10}{8}{3}}
%
% typesets a tableau with 10 boxes in the 1st row, 8 in the 2nd
% and 3 in the 3rd. Curly brackets can be omitted if numbers
% are less than 10.

\newcommand\tableau[1]{%
\def\stest{ }
\def\Ktest{\^}
\acchspace=0ex
\accvspace=0ex
\maxw=0ex
\hbox{%
\@tabforc\thisel{#1}{%
 \ifx\thisel\Ktest{}
 \else
     \repcnt=\thisel%
     \loop{}%
     \advance\acchspace by-\ytsize%
     \raisebox{\accvspace}{\vbox to \ytsize{\hrule \hbox to%
			\ytsize{\vrule height \ytsize\hss}\vss\phrule{\ytsize}}}%
     \advance\repcnt by -1\ifnum\repcnt>1{}\repeat%
     \ifnum\maxw=0\maxw=\acchspace\fi%
     \raisebox{\accvspace}{\vbox to \ytsize{\hbox to 0pt{\vrule height \ytsize\hss}}}%
     \kern\acchspace\acchspace=0ex%
	  \advance\accvspace by -\ytsize%
 \fi}\kern-\maxw}}
 
