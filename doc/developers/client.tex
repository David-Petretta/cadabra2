
\section{Client architecture}

All clients use two threads. One is the main thread and runs most of
the logic in the {\tt DocumentThread} object. The other one is a
thread which communicates with the server process; code for this
thread is in the {\tt ComputeThread} object.  One typically subclasses
the former into a full-fledged object driving the graphical user
interface.

All functionality that deals with GUI updates is isolated in {\tt
  GUIBase}. A client should derive from this abstract base class and
implement the methods there. Clients should also derive from {\tt
  DocumentThread}, but there are no abstract virtual members to
implemented from this base class.

All actual updates to the document stored in the \verb|dtree| member
of \verb|DocumentThread| are made on the GUI thread as well. The
compute thread merely puts requests to change the document onto a
stack, and then requests that the GUI updates the document.
