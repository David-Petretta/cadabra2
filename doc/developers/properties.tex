
\section{Properties}

\subsection{Arguments and validation}

Properties can have arguments. These are not parsed by Python, but
rather by the C++ side. There is a number of reasons for doing things
this way. The most important of them is that this makes it a lot
easier for Cadabra to provide useful feedback on parameters which are
not valid.

Parsing is done by implementing the virtual function
`\verb|parse(const Properties&, keyval_t&)|'. The argument is a
container class which represents the arguments passed to the property
as key/value pairs (the \verb|keyval_t| type is defined in
\verb|Props.hh|).

Properties will be asked to check that they can be associated to a
given pattern through the virtual
`\verb|validate(const Properties&, const exptree&)|' function. The
default implementation returns true for any pattern.

FIXME: the above two need to be merged, because parse may need access
to the actual pattern tree, and once we are there, we may as well
do checking.
