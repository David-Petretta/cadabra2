
\section{Display}

Should properties know about how to display objects associated to
them? Or is that too closely tied, and leading to problems when there
is more than one property attached? How does LaTeXForm fit in?

\subsection{Python output intricacies}

The \verb|print| function always calls the \verb|str| member on
objects to be printed. This \verb|str| function is required to produce
output which looks readable but is also still valid input. In order to
produce proper \LaTeX{}~output, this is therefore not the right
function to use.

Sympy introduces a special printing function \verb|pprint| which will
look at the sympy output settings and call the required display
functions, e.g.~\verb|__latex__| or \verb|__mathml__|. Since we
\emph{always} want notebook cells to produce \LaTeX{} output, this
function is one level too high. Instead, what the Cadabra server uses
is \verb|latex|. That is to say, if the last character of a line is
\verb|;|, it will wrap that line in \verb|latex(...)|.

The only problem that then remains is that we need to be able to
overload that Sympy \verb|latex| function such that it prints our
Cadabra expression objects correctly. But that could be a matter of
simply declaring \verb|__latex__| on \verb|Ex| objects, no?
