
- All algorithms now have uniform and consistent naming. In
particular, it is now \verb|flatten_sum|, \verb|sort_product| and so
on, in verb-noun form. Abbreviations are avoided, so it is now
\verb|eliminate_kronecker|. When the algorithm is to apply a
particular rule or algorithm and there is no better expression for
this, the algorithm is named after this rule, e.g.~\verb|product_rule|.

- Rules for display of output are somewhat different from before. All
property declarations and expressions will return a property or
expression object respectively, which prints in a standard form.
[FIXME! fontend can read ':' and ';'].

- The \verb|collect_terms| function is called automatically after
every change to a sum, there is no need anymore to call it by hand.
The exception is sums in the input of expressions.

- The last entered expression can be recalled with the `\verb|_|'
symbol; this used to be `\verb|%|' but the latter already has a
meaning in Python. 

- All algorithms by default try to act at every node in the tree, in
what is now called `deep' mode. This is equivalent to using the old
`!' notation: \verb|sort_product(_)| now is equivalent to the old
\verb|prodsort!(%)|. If you want to only act at the top level, use
\verb|sort_product(_, deep=False)|.

- 

