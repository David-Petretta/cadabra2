\documentclass[11pt]{article}
\usepackage{charter}
\usepackage[a4paper]{geometry}

\begin{document}
\begin{flushright}
\today
\end{flushright}

\noindent {\bf\Large Cadabra II EPSRC proposal}\\[1ex]
\noindent {\bf\large Dr.~Kasper Peeters}
\vspace{2cm}

\noindent {\bf\large Abstract:}

Cadabra is a Computer Algebra System for the solution of problems in
tensor field theory, in use

Computer Algebra Systems allow mathematicians and physicists to do
tedious computations by computer, which would otherwise take a lot of
time to do with pen and paper. Most existing systems suffer from an
unfortunate abuse of existing programming languages to write down the
mathematical problem. 


\section{Introduction and project outline}A

 major obstacle that prevents the uptake of computer algebra tools by
physicists and mathematicians is the fact that these systems often
speak a language which is far removed from the mathematical language
used to solve problems with pen and paper. One area of mathematics for
which this is particularly manifest is that of tensor field theory,
which is relevant for e.g.~high-energy physics and condensed matter
physics but also for e.g.~machine learning. Often, the task of
transcribing the problem to the programming language used by the
computer algebra system requires substantial programming skills from
the user, which they often do not have and are also not prepared to
learn.

At an deeper and more abstract level, these problems show that it does
often not make sense to use a programming language to describe a
mathematics problem. This is in a way totally unsurprising: most
people do not use a programming language to write letters, so why
would they have to use a programming language to write mathematics?

Such a mixed approach causes harm in the other direction as well:
often, the requirement that language


   - existing cas often neither programmer friendly nor physicist
     friendly: they impose a very basic programming model which
     ignores decades of good practise, but on the other hand are
     not close enough to the language physicists use on paper to do
     computations.

Cadabra was one of the first to recognise these problems and do
something concrete about them. It is unique in the aspects outlined
above. It has has played a role in the publication of nearly 100
research papers. However, it was designed without much emphasis on
portability to non-Linux platforms, and the code is difficult to
approach for others than the lead author.

This means that important new algorithms which have been discovered
more recently do not get implemented. 

The current proposal aims to 

\section{Cadabra users}

16 responses.
- application areas:
    - qft
    - computer vision
    - condensed matter?

benefits seen by users
  - input/output language (111)
  - general applicability, arbitrary dimensions (2)

\section{Project outline}

Problems with the current version, which are also highlighted by the
response to the user questionnaire, are
\begin{itemize}
\item There are no programming constructs 
\item It is difficult or near impossible to write general purpose
  libraries that make use of the core.
\item There is no interface to a scalar computer algebra system.
\item It is hard to build on OS X.
\end{itemize}


\subsection{Help from the SSI}

  - os x gui
  - networking
  - community building tools (have many more users than I am aware of)
  - cmake build system 
  - perhaps organise workshop (1)
  - ipython (11

   
-----
- only cas system with separate data / programming language
- different model for manipulating expressions based on take\_match
- different logic of cells exporting results to subsequent cells.
- 
- native instead of web based

- superset of ipython protocol

- scoping in gui (block groups that share context)
- language mixing and server-based language guessing.



constraints
  - users are not programmers (11111)



compare with
  - maxima
  - ipython
     - much more a programmers workbench than a physicist's tool

  - sympy
     - 

  - sage
  - mathematica xAct/maple other closed-source

\end{document}
