\documentclass[11pt]{book}
\usepackage[scale=.7]{geometry}
\usepackage{charter}
\usepackage[backend=biber, sorting=none]{biblatex}
\usepackage{textfit}
\usepackage{relsize}
\usepackage{setspace}
\usepackage{fancyhdr}
\usepackage{ytableau}
\usepackage{listings}
\usepackage[fleqn]{amsmath}
\usepackage{color}
\usepackage{changepage}
\usepackage[colorlinks=true, linkcolor=red, citecolor=red, urlcolor=black, plainpages=false, pdfpagelabels]{hyperref}
\usepackage{etoolbox}
\usepackage{amssymb}
\usepackage[parfill]{parskip}
\usepackage{graphicx}
\usepackage{wallpaper}
\def\specialcolon{\mathrel{\mathop{:}}\hspace{-.5em}}
\renewcommand{\bar}[1]{\overline{#1}}
%\newcommand{\algorithm}[2]{{\tt\Large\detokenize{#1}}\\[1ex]
%  {\emph{#2}}\\[1ex]
%}
\usepackage{cprotect}
\newcommand{\algorithm}[2]{\subsection{\texttt{#1}}
{\emph{#2}}\\
}
\newcommand{\property}[2]{{\tt\Large\detokenize{#1}}\\[1ex]
{\emph{#2}}\\[1ex]
}
\newcommand{\algo}[1]{{\tt \detokenize{#1}}}
\newcommand{\prop}[1]{{\tt \detokenize{#1}}}
\renewcommand{\author}[1]{{\bfseries #1}}
\newcommand{\email}[1]{, {\tt #1}}

% Math expressions wrapped in \brwrap will get typeset with
% round brackets around them, which have the appropriate size.
% The expression itself can still be broken over multiple lines.

\newcommand\brwrap[3]{%
  \setbox0=\hbox{$#2$}
  \left#1\vbox to \the\ht0{\hbox to 0pt{}}\right.\kern-.2em
  \begingroup #2\endgroup\kern-.15em
  \left.\vbox to \the\ht0{\hbox to 0pt{}}\right#3
}

\renewcommand{\arraystretch}{1.2}
\tolerance=10000
\relpenalty=10
\binoppenalty=10
\hyphenpenalty=10
%\raggedright

%\usepackage[most]{tcolorbox}

\definecolor{backcolour}{rgb}{0.95,0.95,0.92}
\lstnewenvironment{python}[1][]{\lstset{language=python,
   columns=fullflexible,
   xleftmargin=1em,
   belowskip=0pt,
%   aboveskip=5pt,
   tabsize=3,
   backgroundcolor=\color{backcolour},   
   commentstyle={}, % otherwise {#} cadabra arguments look ugly
   breaklines=true,   
   basicstyle=\small\ttfamily\color{blue},
   keywordstyle={}
}}{}
  

\everymath{\displaystyle}

% Page numbers
\pagestyle{fancy}
\fancyhf{} % clear all header and footer fields
\renewcommand{\headrulewidth}{0pt}
\renewcommand{\footrulewidth}{0pt}
\fancyfoot[LE,RO]{{\small\thepage}}
\fancyfoot[LO,RE]{}

% \makeatletter\def\old@comma{,}\catcode`\,=13 \def,{%
% \ifmmode\old@comma\discretionary{}{}{}\else\old@comma\fi}\makeatother

% Ensure that maths broken over multiple lines has a bit of spacing
% between lines.
\lineskiplimit=0mm
\lineskip=1.5ex

\ytableausetup{centertableaux} % smalltableaux

% Chapter and section headers.
\usepackage[compact]{titlesec}

\newcommand{\chapterformat}[1]{\LARGE\larger\bfseries \hspace{3mm}#1}

\titleformat{\chapter}[block]
  {\large\bfseries}{}{0pt}{\hfill{\color[rgb]{.5,.5,.5}{%
  \bfseries\scaletoheight{2cm}{\thechapter}}}%
\\\hbox{~}\hfill\chapterformat}



\bibliography{the_cadabra_book}


\begin{document}
\pagestyle{empty}
\ThisLRCornerWallPaper{1.0}{book_cover}
\begin{flushright}
\today
\end{flushright}
\vspace{6ex}
{\bf \scaletowidth{15cm}{The Cadabra Book}}~\\[.8ex] 
{\it\large A field-theory motivated approach to symbolic computer algebra}\\[6ex]
{\bf \scaletowidth{5cm}{Kasper Peeters}}
\newpage
\pagestyle{empty}
\hbox{~}
\noindent {\smaller This book is available under the terms of the GNU Free
Documentation License, version~1.2.\\
The Cadabra software is available under the terms of the GNU
General Public License, version~3.}
\vfill
\noindent {\large\bf Copyright \copyright~2001-2023 ~Kasper Peeters}\\[3ex]
\href{mailto:kasper.peeters@cadabra.science}{kasper.peeters@cadabra.science}

\tableofcontents

\pagestyle{fancy} 

\chapter{Introduction and overview}
\section{Bird's eye overview}

Cadabra is a symbolic computer algebra system (CAS) designed to solve
problems in physics, in particular (but not limited to) those which
deal with classical and quantum field theory. Its input format is a
subset of TeX, which means that throughout your computations the maths
will all stay in a form which is hopefully already familiar to you.
It has a large number of facilities which make it easy to work with
tensors, anti-commuting objects, implicit indices, implicit coordinate
dependence and so on; things which help to keep mathematical
expressions compact and readable, so that your notebooks resemble what
you would do with pen-and-paper.

% Concrete types of problems you can solve include:
% \begin{itemize}
%   \item 
% \end{itemize}

Cadabra is at its core a Python module (written in C++ and exposed to
Python using pybind11), which also contains a pre-parser which turns
Cadabra's language into pure Python. You can use through a command
line client, a graphical notebook interface, or via Jupyter through
the Cadabra kernel. You can, if you want to, also use it from Python
directly, or as a C++ library.


\section{Cadabra's design philosophy}

Cadabra is built around the fact that many computations do not have
one single and unique path between the starting point and the end
result. When we do computations on paper, we often take bits of an
expression apart, do some manipulations on them, stick them back into
the main expression, and so on. Often, the manipulations that we do
are far from uniquely determined by the problem, and often there is no
way even in principle for a computer to figure out what is 'the best'
thing to do.

What we need the computer to do, in such a case, is to be good at
performing simple but tedious steps, without enforcing on the user how
to do a particular computation. In other words, we want the computer
algebra system to be a scratchpad, leaving us in control of which
steps to take, not forcing us to return to a 'canonical' expression at
every stage.

Most existing computer algebra systems allow for this kind of work
flow only by requiring to stick clumsy 'inert' or 'hold' arguments
onto expressions, by default always 'simplifying' every input to some
form they think is best. Cadabra starts from the other end of the
spectrum, and as a general rule keeps your expression untouched,
unless you explicitly ask for something to be done to it.

Another key issue in the design of symbolic computer algebra systems
has always been whether or not there should be a distinction between
the 'data language' (the language used to write down mathematical
expressions), the 'manipulation language' (the language used to write
down what you want to do with those expressions) and the
'implementation language' (the language used to implement algorithms
which act on mathematical expressions). Many computer algebra systems
take the approach in which these languages are the same (Axiom,
Reduce, Sympy) or mostly the same apart from a small core which uses a
different implementation language (Mathematica, Maple). The Cadabra
project is rooted in the idea that for many applications, it is better
to keep a clean distinction between these three languages. Cadabra
writes mathematics using LaTeX, is programmable in Python, and is
under the hood largely written in C++.

\section{History}

Cadabra was originally written around 2001 to solve a number of
problems related to higher-derivative
supergravity~\cite{Peeters:2000qj,Peeters:2003pv}. It was then
expanded and polished, and first saw its public release in
2007~\cite{kas_cdb_hep}.  During the years that followed, it became
clear that several design decisions were not ideal, such as the use of
a custom programming language and the lack of functionality for
component computations. Over the course of 2015-2016 a large rewrite
took place, which resulted in Cadabra 2.x~\cite{Peeters:2018dyg}.
This new version is programmable in Python and does both abstract and
component computations. From 2017 to 2022 Dominic Price joined the
team. He was responsible for many improvements, such as the add-on
package system, the conversion to pybind11 for the Python bindings,
the implementation of the {\tt meld} algorithm and many others. From
2022 onwards the emphasis has been on making it easier to run Cadabra
(for instance by the introduction of a publically available Jupyter
server with the Cadabra kernel, and by making it much easier to
install the software locally on all platforms). From 2024 the focus is
again on adding more physics functionality as well as tutorials.


%==================================================================
\chapter{The input format}
\input{input_format.tex}
\input{ref_printing.tex}
\input{ref_properties.tex}
\input{ref_indices.tex}
\input{ref_implicit_versus_explicit.tex}

%==================================================================
\chapter{Mathematical properties}
\input{ref_derivatives.tex}

%==================================================================
\chapter{Manipulating expressions}
\input{ref_selecting.tex}
\input{ref_import.tex}
\input{ref_default_simplification}
\input{ref_patterns}
\input{ref_numerical}
\input{ref_dynamical_updates}

%==================================================================
\chapter{Writing your own packages}
\input{ref_programming}
\input{ref_c++_library}

%==================================================================
\chapter{Algorithms}

\section{Substitution and variation}

\cdbalgorithm{distribute}{}

Rewrite a product of sums as a sum of products, as in
\begin{equation}
a\,(b+c) \rightarrow a\,b + a\,c\, .
\end{equation}
This would read
\begin{screen}{1,2}
a (b+c);
@distribute!(%);
a b + a c;
\end{screen}
The algorithm in fact works on all objects which carry
the \subsprop{Distributable} property, 
\begin{screen}{1,2,3}
Op{#}::Distributable.
Op(A+B);
@distribute!(%);
Op(A) + Op(B);
\end{screen}
The primary example of a property which inherits
the \subsprop{Distributable} property
is \subsprop{PartialDerivative}. The \subscommand{distribute} algorithm
thus also automatically writes out partial derivatives of sums as sums
of partial derivatives,
\begin{screen}{1,2,3}
\partial{#}::PartialDerivative.
\partial_{m}(A + B + C):
@distribute!(%);
\partial_{m}(A) + \partial_{m}(B) + \partial_{m}(C);
\end{screen}
~

\cdbseeprop{Distributable}
\cdbseeprop{PartialDerivative}


\input{product_rule}
\cdbalgorithm{substitute}{}

 \label{loc_substitute} Generic substitution command.
Takes a rule or a set of rules according to which an expression
should be modified. If more than one rule is given, it tries each rule
in turn, until the first working one is encountered, after which it
continues with the next node.
\begin{screen}{1,2}
G_{mu nu rho} + F_{mu nu rho};
@substitute!(%)( F_{mu nu rho} -> A_{mu nu} B_{rho} );
G_{mu nu rho} + A_{mu nu} B_{rho};
\end{screen}
\begin{screen}{1,2}
A_{mu nu} B_{nu rho} C_{rho sigma};
@substitute!(%)( A_{m n} C_{p q} -> D_{m q} );
D_{mu sigma} B_{nu rho};
\end{screen}
This command takes full care of dummy index relabelling, as the
following example shows:
\begin{screen}{1,2,3}
{m,n,q,d1,d2,d3,d4}::Indices(vector).
a_{m} b_{n};
@substitute!(%)( a_{q} -> c_{m n} d_{m n q} );
c_{d1 d2} * d_{d1 d2 m} * b_{n};
\end{screen}
By postfixing a name with a question mark, it becomes a pattern.

The substitution algorithm can do very complicated things; for more
detailed information on substitution, see the manual.

\cdbseealgo{vary}

\cdbalgorithm{vary}{}

Generic variation command.  Takes a rule or a set of rules
according to which the terms in a sum should be varied. In every term,
apply the rule once to every factor.
\begin{screen}{1,2}
A B + A C;
@vary(%)( A -> \epsilon D,
          B -> \epsilon C,
          C -> \epsilon A - \epsilon B );
\epsilon D B + A \epsilon C + \epsilon D C 
                   + A (\epsilon A - \epsilon B);
\end{screen}
It also works when acting on powers, in which case it will use the
product rule to expand them.
\begin{screen}{1,2}
A**3;
@vary(%)( A -> \delta{A} );
3 A**{2} \delta{A};
\end{screen}
This algorithm is thus mostly intended for variational derivatives
(subsequent partial integrations can be done
using \subscommand{pintegrate}). 

Note: In the examples above, the command is applied only at the top
level (there is no exclamation mark used in the call
of \subscommand{vary}). To understand why this is important, compare
the following two examples. The first one works as expected,
\begin{screen}{1,2}
A B;
@vary(%)( A -> a, B -> b);
a * B + A * b;
\end{screen}
In the second one, we add an exclamation mark,
\begin{screen}{1,2}
A B;
@vary!(%)( A -> a, B -> b);
0;
\end{screen}
The reason why we now get a zero is that in the first step, the vary
command acts in each of the individual factors, producing
\begin{screen}{0}
a b;
\end{screen}
It then acts once more at the level of the product. But now there are
no uppercase symbols left anymore, and the variation produces zero.

\cdbseealgo{pintegrate}
\cdbseealgo{substitute}


\cdbalgorithm{expand\_power}{}

Expand powers into repeated products, i.e.~do the opposite
of \subscommand{collect\_factors}. For example,
\begin{screen}{1,2,4,5,7}
(A B)**3:
@expand_power!(%);
(A * B) * (A * B) * (A * B);
@prodflatten!(%):
@prodsort!(%);
A A A B B B;
@collect_factors!(%);
A**3 * B**3;
\end{screen}
This command automatically takes care of index relabelling when
necessary, as in the following example,
\begin{screen}{1,2,3,4}
{m,n,p,q,r}::Indices(vector).
(A_m B_m)**3:
@expand_power!(%):
@prodflatten!(%);
A_{m} * B_{m} * A_{n} * B_{n} * A_{p} * B_{p};
\end{screen}
~

\cdbseealgo{collect_factors}


\cdbalgorithm{unwrap}{}

Move objects out of \subsprop{Derivative}s or \subsprop{Accent}s when
they do not depend on these operators. 

Accents will get removed from objects which do not depend on them, as
in the following example:
\begin{screen}{1,2,3,5,6,8}
\hat{#}::Accent.
\hat{#}::Distributable.
B::Depends(\hat).

\hat{A+B+C}:
@distribute!(%);
\hat{A} + \hat{B} + \hat{C};
@unwrap!(%);
A + \hat{B} + C;
\end{screen}

Derivatives will be set to zero if an object inside does not depend on
it. All objects which are annihilated by the derivative operator are
moved to the front (taking into account anti-commutativity if necessary),
\begin{screen}{1,2,4,5}
\partial{#}::PartialDerivative.
{A,B,C,D}::AntiCommuting.
x::Coordinate.
{B,D}::Depends(\partial).

\partial_{x}( A B C D ):
@unwrap!(%);
- A C \partial_{x}{B D};
\end{screen}
Note that a product remains inside the derivative; to expand that
use \subscommand{prodrule}. 

Here is another example
\begin{screen}{1,2,3,4,6}
\del{#}::Derivative.
X::Depends(\del).
\del{X*Y*Z}:
@prodrule!(%);
\del{X} * Y * Z + X * \del{Y} * Z + X * Y * \del{Z};
@unwrap!(%);
\del{X}*Y*Z;
\end{screen}

Note that all objects are by default constants for the action of
\subsprop{Derivative} operators. If you want objects to stay inside
derivative operators you have to explicitly declare that they depend
on the derivative operator or on the coordinate with respect to which
you take a derivative.

\cdbseeprop{Accent}
\cdbseeprop{Derivative}
\cdbseeprop{PartialDerivative}
\cdbseealgo{prodrule}

\input{integrate_by_parts}

\section{Metrics and bundles}

\cdbalgorithm{eliminate\_kr}{}

Eliminates Kronecker delta symbols by performing index
contractions. Also replaces contracted Kronecker delta symbols with
the range over which the index runs, if known. 
\begin{screen}{1,2,3}
\delta_{m n}::KroneckerDelta.
A_{m p} \delta_{p q} B_{q n};
@eliminate_kr!(%);
A_{m q} B_{q n};
\end{screen}
The index range is set as usual with \subsprop{Integer},
\begin{screen}{1,2,3,4}
{m,n,p,q}::Integer(0..d-1).
\delta_{m n}::KroneckerDelta.
\delta_{p q} \delta_{p q};
@eliminate_kr!(%);
d;
\end{screen}
In order to eliminate metric factors (i.e.~to `raise' and 'lower'
indices) use the algorithm \subscommand{eliminate\_metric}.

\cdbseeprop{Integer}
\cdbseealgo{eliminate_metric}

\cdbalgorithm{eliminate\_metric}{}

Eliminates metric and inverse metric objects.
\begin{screen}{1,2,3,4,5,6,7,8,10}
{m, n, p, q, r}::Indices(vector, position=fixed).
{m, n, p, q, r}::Integer(0..9).
g_{m n}::Metric.
g^{m n}::InverseMetric.
g_{m}^{n}::KroneckerDelta.
g^{m}_{n}::KroneckerDelta.
g_{m p} g^{p m};
@eliminate_metric!(%);
g^{p}_{p};
@eliminate_kr!(%);
10;
\end{screen}
Other elimination commands of a similar type are
\subscommand{eliminate\_kr} for Kronecker delta symbols
and \subscommand{eliminate\_vielbein} for vielbeine.

\cdbseealgo{eliminate_kr}
\cdbseealgo{eliminate_vielbein}

\cdbalgorithm{eliminate\_vielbein}{}

Eliminates vielbein objects.
\begin{screen}{1,2,3,4,5}
{ m, n, p }::Indices(flat).
{ \mu, \nu, \rho }::Indices(curved).
e_{m \mu}::Vielbein.
H_{m n p} e_{m \mu};
@eliminate_vielbein!(%){H_{\mu\nu\rho}};
H_{\mu n p};
\end{screen}
Other elimination commands of a similar type are
\subscommand{eliminate\_kr} for Kronecker delta symbols
and \subscommand{eliminate\_metric} for metric and inverse metric objects.

\cdbseealgo{eliminate_kr}
\cdbseealgo{eliminate_metric}

\cdbalgorithm{einsteinify}{}

In an expression containing dummy indices at the same position
(i.e.~either both subscripts or both superscripts), raise one of the
indices.
\begin{screen}{1,2}
A_{m} A_{m};
@einsteinify!(%);
A^{m} A_{m};
\end{screen}
If an additional argument is given to this command, it instead inserts
``inverse metric'' objects, with the name as indicated by the
additional argument.
\begin{screen}{1,2}
{m,n}::Indices.
A_{m} A_{m};
@einsteinify!(%){\eta};
A_{m} A_{n} \eta^{m n};
\end{screen}
Note that the second form requires that there are enough dummy indices
defined through the use of \subsprop{Indices}.

\cdbseealgo{eliminate_kr}
\cdbseeprop{Indices}

\input{epsilon_to_delta}
\input{expand_delta}
\input{reduce_delta}

\section{Index manipulations}

\cdbalgorithm{combine}{}

Combine two consecutive objects with indexbrackets and consecutive
contracted indices into one object with an indexbracket. An example
with two contracted matrices:
\begin{screen}{1,2}
(\Gamma_r)_{\alpha\beta} (\Gamma_{s t u})_{\beta\gamma};
@combine!(%);
(\Gamma_r \Gamma_{s t u})_{\alpha\gamma};
\end{screen}
An example with a matrix and a vector:
\begin{screen}{1,2}
(\Gamma_r)_{\alpha\beta} v_{\beta};
@combine!(%);
(\Gamma_{r} v)_{\alpha};
\end{screen}
The inverse is done by \subscommand{expand}.

\cdbseealgo{expand}

\input{explicit_indices}
\input{lower_free_indices}
\input{raise_free_indices}
\cdbalgorithm{split\_index}{}

Replace a sum by a sum-of-sums, abstractly. Concretely, replaces all
index contractions of a given type by a sum of two terms, each with
indices of a different type. Useful for Kaluza-Klein reductions and
the like. An example makes this more clear:
\begin{screen}{1,2,3,4,5,7,8}
{M,N,P,Q,R}::Indices(full).
{m,n,p,q,r}::Indices(space1).
{a,b,c,d,e}::Indices(space2).
A_{M p} B_{M p};
@split_index(%){M,m,a};
A_{m p} B_{m p} + A_{a p} B_{a p};
@pop(%);
@split_index(%){M,m,4};
A_{m p} B_{m p} + A_{4 p} B_{4 p};
\end{screen}
Note that the two index types into wich the original indices should be
split can be either symbolic (as in the first case above) or numeric
(as in the second case).

\cdbseeprop{Indices}
\cdbseealgo{rewrite_indices}

\input{untrace}
\cdbalgorithm{rename\_dummies}{}

Rename the dummy indices in an expression. This can be necessary in
order to make various terms in a sum use the same names for the
indices, so that they can be collected.
\begin{screen}{1,2,3,5}
{m,n,p,q,r,s}::Indices(vector).
A_{m n} B_{m n} - A_{p q} B_{p q};
@rename_dummies!(%);
A_{m n} B_{m n} - A_{m n} B_{m n};
@collect_terms!(%);
0;
\end{screen}
Note that the indices need to have been declared as being part of an
index list, using the \subsprop{Indices} property.

\cdbseeprop{Indices}

\cdbalgorithm{rewrite\_indices}{}

Rewrite indices on an object by contracting it with a second object
which contains indices of both the old and the new type (a vielbein,
in other words, or a metric). A vielbein example is
\begin{screen}{1,2,3,4}
{m,n,p}::Indices(flat).
{\mu,\nu,\rho}::Indices(curved).
T_{m n p};
@rewrite_indices!(%){ T_{\mu\nu\rho} }{ e_{\mu}^{n} };
T_{\mu \nu \rho} e_{\mu}^{m} e_{\nu}^{n} e_{\rho}^{p};
\end{screen}
If you want to raise or lower an index with a metric, this can also be
done with as an index rewriting command, as the following example shows:
\begin{screen}{1,2,3}
{m,n,p,q,r,s}::Indices(curved, position=fixed).
H_{m n p};
@rewrite_indices!(%){ H^{m n p} }{ g_{m n} };
H^{q r s} g_{m q} g_{n r} g_{p s};
\end{screen}
As these examples show, the desired form of the tensor should be given
as the first argument, and the conversion object (metric, vielbein) as
the second object. 

\cdbseealgo{split_index}

\cdbalgorithm{expand}{}

Write out products of matrices and vectors inside indexbrackets,
inserting new dummy indices for the contraction. This requires that
the objects inside the index bracket are properly declared to
have \subsprop{Matrix} or \subsprop{ImplicitIndex} properties.  

Here is an example with multiple matrices:
\begin{screen}{1,2,3}
{a,b,c,d,e}::Indices.
{A,B,C,D}::Matrix.
(A B C D)_{a b};
@expand!(%);
A_{a c} B_{c d} C_{d e} D_{e b};
\end{screen}
Compare the above to the following example, in which one of the
objects inside the bracket is no longer a matrix:
\begin{screen}{1,2,3}
{a,b,c,d,e}::Indices.
{A,B,D}::Matrix.
(A B C D)_{a b};
@expand!(%);
A_{a c} B_{c d} C D_{d b};
\end{screen}
Finally, an example with matrices carrying additional labels, as well
as a vector object:
\begin{screen}{1,2}
{\alpha,\beta}::Indices.
\Gamma{#}::Matrix.
v::ImplicitIndex.
(\Gamma_{r} v)_{\alpha};
@expand!(%);
(\Gamma_{r})_{\alpha \beta} v_{\beta};
\end{screen}
Note that in all cases, the indices on the indexbracket have to be
part of a large enough set so that the dummy indices can be generated.

\cdbseeprop{Matrix}
\cdbseeprop{ImplicitIndex}
\cdbseeprop{Indices}
\cdbseealgo{combine}


\section{Tensor component values}

\input{complete}
\input{evaluate}

\section{Factorisation}

\cdbalgorithm{factor\_in}{}

Given a list of symbols, this algorithm collects terms in a sum that
only differ by pre-factors consisting of these given symbols. As an
example,
\begin{screen}{1,2}
a b + a c + a d:
@factor_in!(%){b,c};
(b + c) a + a d;
\end{screen}
The name is perhaps most easily understood by thinking of it as a
complement to {\tt factor\_out}. Or in case you are familiar with
FORM, {\tt factor\_in} is like its {\tt antibracket} statement.

The algorithm of course also works with indexed objects, as in
\begin{screen}{1,2}
A_{m} B_{m} + C_{m} A_{m};
@factor_in!(%){B_{n},C_{n}};
(B_{m} + C_{m}) A_{m};
\end{screen}
(this is still work in progress).

\cdbseealgo{factor_out}

\cdbalgorithm{factor\_out}{}

Given a list of symbols, this algorithm tried to factor those symbols
out of terms. As an example,
\begin{screen}{1,2}
a b + a c e + a d:
@factor_out!(%){a};
a ( b + c e + d );
\end{screen}
In case you are familiar with FORM, {\tt factor\_out} is like its {\tt
bracket} statement. If you add more factors to factor out, it works as
in
\begin{screen}{1,2}
a b + a c e + a c + c e + c d + a d:
@factor_out!(%){a,c};
a (b + d) + c ( e + d ) + a c (e + 1);
\end{screen}
That is, separate terms will be generated for terms which differ by
powers of the factors to be factored out.

The algorithm of course also works with indexed objects, as in
\begin{screen}{1,2}
A_{m} B_{m} + C_{m} A_{m};
@factor_out!(%){A_{m}};
A_{m} (B_{m} + C_{m}) ;
\end{screen}
(this is still work in progress).

\cdbseealgo{factor_in}


\section{Spinors and fermions}

\cdbalgorithm{expand\_diracbar}{}

Rewrite the Dirac conjugate of a product of spinors and gamma matrices
as a product of Dirac and hermitean conjugates. This uses
\begin{equation} 
\bar\psi = i \psi^\dagger\Gamma^0\,,
\end{equation}
together with 
\begin{equation}
\Gamma_m^{\dagger} = \Gamma_0\Gamma_m\Gamma_0 \,.
\end{equation}
For example,
\begin{screen}{1,2,3,4,5}
\bar{#}::DiracBar.
\psi::Spinor(dimension=10).
\Gamma{#}::GammaMatrix.
\bar{\Gamma^{m n p} \psi};
@rewrite_diracbar!(%);
\bar{\psi} \Gamma^{m n p};
\end{screen}
~

\cdbseealgo{spinorsort}
\cdbseeprop{GammaMatrix}
\cdbseeprop{DiracBar}
\cdbseeprop{Spinor}

\cdbalgorithm{fierz}{}

Perform a Fierz transformation on a product of four spinors. 
\begin{screen}{1,2}
{\theta,\chi,\psi,\lambda}::Spinor(dimension=4, type=Majorana).
{m,n,p#}::Indices.
{m,n,p}::Integer(0..3).
\Gamma{#}::GammaMatrix.
\bar{#}::DiracBar.
\bar{\theta} \Gamma_{m} \chi \bar{\psi} \Gamma^{m} \lambda;
@fierz!(%)(\theta, \lambda, \psi, \chi);
\end{screen}
The argument to \subscommand{fierz} is the required order of the
fermions; note that this algorithm does not flip around Majorana
spinors and \subscommand{spinorsort} should be used for that.  Also
note that it is important to define not only the symbols representing
the spinors, Dirac bar and gamma matrices, but also the range of the
indices.

\cdbseealgo{spinorsort}
\cdbseeprop{GammaMatrix}
\cdbseeprop{DiracBar}
\cdbseeprop{Spinor}

\cdbalgorithm{join}{}

Join two fully anti-symmetrised gamma matrix
products according to the expression
\begin{equation}
   \Gamma^{b_{1}\dots b_{n}}\Gamma_{a_{1}\dots a_{m}} =
      \sum_{p=0}^{\text{min}(n,m)}\ \frac{n! m!}{(n-p)! (m-p)! p!}
         \Gamma^{[b_{1}\ldots b_{n-p}}{}_{[a_{p+1}\ldots a_{m}}
         \eta^{b_{n-p+1}\ldots b_{n}]}{}_{a_{1}\ldots a_{m-p}]} \, .
\end{equation}
This is the opposite of \subscommand{gammasplit}.

Without further arguments, the anti-symmetrisations will be left
implicit. The argument ``{\tt expand}'' instead performs the sum over
all anti-symmetrisations, which may lead to an enormous number of
terms if the number of indices on the gamma matrices is large. Compare
\begin{screen}{1,2}
\Gamma{#}::GammaMatrix(metric=g).
\Gamma_{m n} \Gamma_{p};
@join!(%);
\Gamma_{m n p} + 2 \Gamma_{m} g_{n p};
\end{screen}
with
\begin{screen}{1,2}
\Gamma{#}::GammaMatrix(metric=g).
\Gamma_{m n} \Gamma_{p};
@join!(%){expand};
\Gamma_{m n p} + \Gamma_{m} g_{n p} - \Gamma_{n} g_{m p};
\end{screen}
Note that the gamma matrices need to have a metric associated to them
in order for this algorithm to work.

In order to reduce the number somewhat, one can instruct the algorithm
to make use of generalised Kronecker delta symbols in the result;
these symbols are defined as
\begin{equation}
\delta^{r_1}{}_{s_1}{}^{r_2}{}_{s_2}\cdots{}^{r_n}{}_{s_n}
= \delta^{[r_1}{}_{s_1}\delta^{r_2}{}_{s_2}\cdots {}^{r_n]}{}_{s_n}\, .
\end{equation}
Anti-symmetrisation is implied in the set of even-numbered
indices. The use of these symbols is triggered by the ``{\tt
gendelta}'' option,
\begin{screen}{1,2}
{m,n,p,q}::Indices(position=fixed).
\Gamma{#}::GammaMatrix(metric=\delta).
\Gamma_{m n} \Gamma^{p q};
@join!(%){expand}{gendelta};
 \Gamma_{m n}^{p q} + \Gamma_{m}^{q} \delta_{n}^{p} 
    - \Gamma_{m}^{p} \delta_{n}^{q} - \Gamma_{n}^{q} \delta_{m}^{p} 
    + \Gamma_{n}^{p} \delta_{m}^{q} + 2 \delta_{n}^{p}_{m}^{q};
\end{screen}

Finally, to select only a single term (for a given $p$) in this
expansion, give the join an argument with the value of $p$. 
\begin{screen}{1,2}
\Gamma{#}::GammaMatrix(metric=g).
\Gamma_{m n} \Gamma_{p};
@join!(%){expand}{3};
\Gamma_{m n p};
\end{screen}
This option can also be combined with {\tt gendelta} if required.

\cdbseeprop{GammaMatrix}
\cdbseeprop{KroneckerDelta}
\cdbseealgo{gammasplit}

\input{sort_spinors}
\input{split_gamma}

\section{Sorting and canonicalisation}
  
\cdbalgorithm{asym}{}

Anti-symmetrise a product or tensor in the indicated objects. This works
both with normal objects, as in
\begin{screen}{1,2}
A B C;
@asym!(%){A,B,C};
1/6 A B C - 1/6 A C B - 1/6 B A C 
          + 1/6 B C A + 1/6 C A B - 1/6 C B A;
\end{screen}
as well as with indices. When used with indices, remember to also
indicate whether you want to symmetrise upper or lower indices, as in
the example below.
\begin{screen}{1,2}
A_{m n} B_{p};
@asym!(%){ _{m}, _{n}, _{p} };
1/6 A_{m n} B_{p} - 1/6 A_{m p} B_{n} - 1/6 A_{n m} B_{p}
     + 1/6 A_{n p} B_{m} + 1/6 A_{p m} B_{n} - 1/6 A_{p n} B_{m};
\end{screen}
Symmetrisation (i.e.~using plus signs for all terms) is handled by
the \subscommand{sym} algorithm.  ~

\cdbseealgo{sym}
\cdbseealgo{young_project}
\cdbseealgo{young_project_tensor}

\cdbalgorithm{canonicalise}{}

\label{loc_canonicalise}
Canonicalise a product of tensors, using the mono-term\index{mono-term
symmetries} index symmetries of the individual tensors and the
exchange symmetries of identical tensors. Tensor exchange takes into
account commutativity properties of identical tensors.

Note that this algorithm does not take into account multi-term
symmetries such as the Ricci identity of the Riemann tensor; those
canonicalisation procedures require the use
of \subscommand{young\_project\_tensor}
or \subscommand{young\_project\_product}. Similarly,
dimension-dependent identities are not taken into account, use
\subscommand{decompose\_product} for those.

In order to specify symmetries of tensors you need to use symmetry
properties such as \subsprop{Symmetric}, \subsprop{AntiSymmetric}
or \subsprop{TableauSymmetry}. The following example illustrates this.
\begin{screen}{1,2,3,4}
A_{m n}::AntiSymmetric.
B_{p q}::Symmetric.
A_{m n} B_{m n};
@canonicalise!(%);
0;
\end{screen}
If the various terms in an expression use different index names, you
may need an additional call to \subscommand{rename\_dummies}
before \subscommand{collect\_terms} is able to collect all terms
together:
\begin{screen}{1,2,3,4,5,7,9}
{m,n,p,q,r,s}::Indices.
A_{m n}::AntiSymmetric.
C_{p q r}::AntiSymmetric.
A_{m n} C_{m n q} + A_{s r} C_{s q r};
@canonicalise!(%);
A_{m n} * C_{q m n} - A_{r s} * C_{q r s};
@rename_dummies!(%);
A_{m n} * C_{q m n} - A_{m n} * C_{q m n};
@collect_terms!(%);
0;
\end{screen}

If you have symmetric or anti-symmetric tensors with many indices, it
sometimes pays off to sort them to the end of the expression (this may
speed up the canonicalisation process considerably).

\cdbseealgo{young_project_tensor}
\cdbseealgo{decompose_product}
\cdbseealgo{rename_dummies}
\cdbseealgo{collect_terms}

\cdbalgorithm{young\_project\_product}{}

Project all tensors in a product with their Young tableau
projector. Each factor is projected in turn, after which the product
is distributed and then canonicalised. This is often faster and more
memory-efficient than first projecting every factor and then
distributing.

Young projection can be used to find equalities between tensor
polynomials which are due to multi-term symmetries, such as the Ricci
identity in the example below.
\begin{screen}{1,2,3,4,5,6}
{a,b,c,d}::Indices.
R_{a b c d}::RiemannTensor.
2 R_{a b c d} R_{a c b d} - R_{a b c d} R_{a b c d};
@young_project_product!(%);
@sumflatten!(%);
@collect_terms!(%);
0;
\end{screen}
~

\cdbseealgo{decompose}
\cdbseealgo{young_project_tensor}

\cdbalgorithm{young\_project\_tensor}{}

Project tensors with their Young projection operator. This works for
simple symmetric or anti-symmetric objects, as in
\begin{screen}{1,2,3}
A_{m n}::Symmetric.
A_{m n} A_{m p};
@young_project_tensor!(%);
(1/2 A_{m n} + 1/2 A_{n m}) (1/2 A_{m p} + 1/2 A_{p m});
\end{screen}
but more generically works for any tensor which has
a \subsprop{TableauSymmetry} property attached to it. 
\begin{screen}{1,3}
A_{m n p}::TableauSymmetry(shape={2,1}, indices={0,2,1}).
A_{m n p};
@young_project_tensor!(%);
1/3 A_{m n p} + 1/3 A_{p n m} - 1/3 A_{n m p} - 1/3 A_{p m n};
\end{screen}
When the argument {\tt ModuloMonoterm} is added, the resulting
expression will be simplified using the monoterm symmetries of the
tensor,
\begin{screen}{1,3}
A_{m n p}::TableauSymmetry(shape={2,1}, indices={0,2,1}).
A_{m n p};
@young_project_tensor!(%){ModuloMonoterm};
2/3 A_{m n p} - 1/3 A_{n p m} + 1/3 A_{m p n};
\end{screen}
(in this example, the tensor is anti-symmetric in the indices~0 and 1,
hence the simplification compared to the previous example).

\cdbseealgo{young_project}
\cdbseealgo{young_project_product}
\cdbseeprop{TableauSymmetry}
\cdbseeprop{Symmetric}
\cdbseeprop{AntiSymmetric}



\input{meld}
\cdbalgorithm{prodsort}{}

Sort factors in a product, taking into account
any \subsprop{SortOrder} properties. Also takes into account
commutativity properties, such as \subsprop{SelfCommuting}. If no sort
order is given, it first does a lexographical sort based on the name
of the factor, and if two names are identical, does a sort based on
the number of children and (if this number is equal) a lexographical
comparison of the names of the children.

The simplest sort is illustrated below,
\begin{screen}{1,2}
C B A D;
@prodsort!(%);
A B C D;
\end{screen}
We can declare the objects to be anti-commuting, which then leads to
\begin{screen}{1,2,3}
{A, B, C, D}::AntiCommuting.
C B A D;
@prodsort!(%);
(-1) A B C D;
\end{screen}
For indexed objects, the anti-commutativity of components is indicated
using the \subsprop{SelfAntiCommuting} property,
\begin{screen}{1,2,3}
\psi_{m}::SelfAntiCommuting.
\psi_{n} \psi_{m} \psi_{p};
@prodsort!(%);
(-1)  \psi_{m} \psi_{n} \psi_{p};
\end{screen}
Finally, the lexographical sort order can be overridden by using
the \subsprop{SortOrder} property,
\begin{screen}{1,2}
{D, C, B, A}::SortOrder.
{A, B, C, D}::AntiCommuting.
C B A D;
@prodsort!(%);
(-1) D C B A;
\end{screen}


%\kcomment{KP}{Warning! Does not yet know about {$\backslash$pow}}

\cdbseeprop{SortOrder}
\cdbseeprop{Commuting}
\cdbseeprop{AntiCommuting}

\input{sort_sum}

\section{Weights and perturbations}

\cdbalgorithm{drop\_weight}{}

Drop those terms for which a product has the indicated
weight. Weights are computed by making use of the \subsprop{Weight}
property of symbols. This algorithm does the opposite
of \subscommand{keep\_weight}.

As an example, consider the simple case in which we want to drop all
terms with 3~fields. This is done using
\begin{screen}{1,2,3}
{A,B}::Weight(label=field);
ex:=A B B + A A A + A B + B:
drop_weight(_, $field=3$);
A B + B;
\end{screen}

However, you can also do more complicated things by assigning non-unit
weights to symbols, as in the example below,
\begin{screen}{1,2,3,4}
{A,B}::Weight(label=field).
C::Weight(label=field, value=2).
A B B + A A A + A B + A C + B:
@drop_weight!(%){field}{3};
A B + B;
\end{screen}

Weights can be ``inherited'' by operators by using
the \subsprop{WeightInherit} property. Here is an example using
partial derivatives,
\begin{screen}{1,2,3,4,6}
{\phi,\chi}::Weight(label=small, value=1).
\partial{#}::PartialDerivative.
\partial{#}::WeightInherit(label=all, type=Multiplicative).
\phi \partial_{0}{\phi} + \partial_{0}{\lambda} 
                                  + \lambda \partial_{3}{\chi}:
@drop_weight!(%){small}{1};
\phi \partial_{0}{\phi} + \partial_{0}{\lambda};
\end{screen}
~

\cdbseealgo{keep_weight}
\cdbseeprop{Weight}
\cdbseeprop{WeightInherit}

\cdbalgorithm{keep\_weight}{}

Keep only those terms for which a product has the indicated
weight. Weights are computed by making use of the \subsprop{Weight}
property of symbols. This algorithm does the opposite
of \subscommand{drop\_weight}.

As an example, consider the simple case in which we want to keep all
terms with 3~fields. This is done using
\begin{screen}{1,2,3}
{A,B}::Weight(label=field).
A B B + A A A + A B + B:
@keep_weight!(%){field}{3};
A B B + A A A 
\end{screen}

However, you can also do more complicated things by assigning non-unit
weights to symbols, as in the example below,
\begin{screen}{1,2,3,4}
{A,B}::Weight(label=field).
C::Weight(label=field, value=2).
A B B + A A A + A B + A C + B:
@keep_weight!(%){field}{3};
A B B + A A A + A C
\end{screen}

Weights also apply to tensorial expressions. Consider e.g.~a situation
in which we have a polynomial of the type
\begin{equation}
c^{a} + c^{a}_{b} x^{b} + c^{a}_{b c} x^{b} x^{c} + c^{a}_{b c d}
x^{b} x^{c} x^{d};
\end{equation}
and we want to keep only the quadratic term. This can be done using
\begin{screen}{1,2,3,4}
x^{a}::Weight(label=crd, value=1).
c^{#}::Weight(label=crd, value=0).
c^{a} + c^{a}_{b} x^{b} + c^{a}_{b c} x^{b} x^{c} + c^{a}_{b c d} x^{b} x^{c} x^{d};
@keep_weight!(%){crd}{2};
c^{a}_{b c} x^{b} x^{c};
\end{screen}

Weights can be ``inherited'' by operators by using
the \subsprop{WeightInherit} property. Here is an example using
partial derivatives,
\begin{screen}{1,2,3,4,6}
{\phi,\chi}::Weight(label=small, value=1).
\partial{#}::PartialDerivative.
\partial{#}::WeightInherit(label=all, type=Multiplicative).
\phi \partial_{0}{\phi} + \partial_{0}{\lambda} 
                               + \lambda \partial_{3}{\chi}:
@keep_weight!(%){small}{1};
\lambda \partial_{3}{\chi};
\end{screen}

If you want to use weights for dimension counting, in which operators
can also carry a dimension themselves (e.g.~derivatives), then use the
\verb|self| attribute,
\begin{screen}{1,2,3,4,6}
{\phi,\chi}::Weight(label=length, value=1).

x::Coordinate.
\partial{#}::PartialDerivative.
\partial{#}::WeightInherit(label=length, type=Multiplicative, self=-1).

\phi \partial_{x}{\phi} + \phi\chi + \partial_{x}{ \phi \chi**2 };
@keep_weight!(%){length}{1};
\end{screen}
This keeps only the first term.

~
\cdbseealgo{drop_weight}
\cdbseeprop{Weight}
\cdbseeprop{WeightInherit}


\section{Simplification}

\cdbalgorithm{collect\_factors}{}

Collect factors in a product that differ only by their exponent. Note
that factors containing sub- or superscripted indices do not get
collected (i.e.~$A_m A^m$ does not get reduced to $(A_m)^2$).
\begin{screen}{1,2}
A A B A B A;
@collect_factors!(%);
A**4 B**2;
\end{screen}
Arbitrary powers can be collected this way,
\begin{screen}{1,2}
X X**(-1) X**(-4);
@collect_factors!(%);
X**(-3);
\end{screen}
The exponent notation can be expanded again 
using \subscommand{expand\_power}.

~

\cdbseealgo{expand_power}
\cdbseealgo{collect_terms}

\cdbalgorithm{collect\_terms}{}

\label{loc_collect_terms}
Collect terms in a sum that differ only by their numerical
pre-factor. This is called automatically on all new input, and also by
some algorithms (in which case it will be indicated in the description
of the command), but in general has to be called by hand.

Note that this command only collects terms which are identical, it
does not collect terms which are different but mathematically
equivalent. See~\subscommand{sumsort} for an example.

\cdbseealgo{collect_factors}
\cdbseealgo{sumsort}

\input{map_sympy}
\input{simplify}

\section{Representations}

\cdbalgorithm{decompose}{}

Decompose a tensor monomial on a given basis of monomials. The basis
should be given in the second argument. All tensor symmetries,
including those implied by Young tableau Garnir symmetries, are taken
into account. Example,
\begin{screen}{0,1,2,3,5,6}
{m,n,p,q}::Indices(vector).
{m,n,p,q}::Integer(0..10).
R_{m n p q}::RiemannTensor.

R_{m n q p} R_{m p n q};
@decompose!(%)( R_{m n p q} R_{m n p q} );
{ -1/2 };
\end{screen}
Note that this algorithm does not yet take into account
dimension-dependent identities, but it is nevertheless already
required that the index range is specified.

\cdbseealgo{young_project_tensor}
\cdbseealgo{canonicalise}
\cdbseeprop{TableauSymmetry}

\cdbalgorithm{decompose\_product}{}

Decompose a product of tensors by writing it out in terms of
irreducible Young tableau representations, discarding the ones which
vanish in the indicated dimension, and putting the results back
together again. This algorithm can thus be used to equate terms which
are identical only in certain dimensions.

If there are no dimension-dependent identities playing a role in the
product, then \subscommand{decompose\_product} returns the original
expression,
\begin{screen}{1,2,3,4,5,6,7}
{ m, n, p, q }::Indices(vector).
{ m, n, p, q }::Integer(1..4).
{ A_{m n p}, B_{m n p} }::AntiSymmetric.
ex:= A_{m n p} B_{m n q} - A_{m n q} B_{m n p}.
decompose_product(_)
A_{m n p} B_{m n q} - A_{m n q} B_{m n p};
\end{screen}
However, in the present example, a Schouten identity makes the
expression vanish identically in three dimensions,
\begin{screen}{1,2,3,4}
{ m, n, p, q }::Integer(1..3).
decompose_product(ex)
0;
\end{screen}

Note that \subscommand{decompose\_product} is unfortunately
computationally expensive, and is therefore not practical for large
dimensions.

\cdbalgorithm{lr\_tensor}{}

Compute the tensor product of two tableaux or filled tableaux. The
algorithm acts on objects which have the \subsprop{Tableau}
or \subsprop{FilledTableau} property, through which it is possible to
set the dimension. The standard Littlewoord-Richardson algorithm is
used to construct the tableaux in the tensor product. An example
with \subsprop{Tableau} objects is given below.
\begin{screen}{1,2,3}
\tableau{#}::Tableau(dimension=10).
\tableau{2}{2} \tableau{2}{2};
@lr_tensor!(%);
\tableau{4 4} + \tableau{4 3 1} + \tableau{4 2 2} 
     + \tableau{3 3 1 1} + \tableau{3 2 2 1} 
     + \tableau{2 2 2 2};
\end{screen}
In the graphical interface the output will show up as proper Young tableaux,
\begin{equation}
\tableau{4 4} + \tableau{4 3 1} + \tableau{4 2 2} + \tableau{3 3 1 1} + \tableau{3 2 2 1} + \tableau{2 2 2 2};
\end{equation}

The same example, but now with \subsprop{FilledTableau} objects, is
\begin{screen}{1,2,3}
\tableau{#}::FilledTableau(dimension=10).
\tableau{0,0}{1,1} \tableau{a,a}{b,b}:
@lr_tensor!(%);
\end{screen}
This will again output a sum of \verb|\tableau| objects. In the graphical
interface they will be typeset as
\begin{equation}
\ftableau{{0}{0}{a}{a},{1}{1}{b}{b}} + \ftableau{{0}{0}{a}{a},{1}{1}{b},{b}} + \ftableau{{0}{0}{a}{a},{1}{1},{b}{b}} + \ftableau{{0}{0}{a},{1}{1}{b},{a},{b}} + \ftableau{{0}{0}{a},{1}{1},{a}{b},{b}} + \ftableau{{0}{0},{1}{1},{a}{a},{b}{b}};
\end{equation}

\cdbseeprop{Tableau}
\cdbseeprop{FilledTableau}



\section{Sub-expression manipulation}

\cdbalgorithm{replace\_match}{}

Replaces a subset of terms in a sum or list which match the given
pattern. This is like a substitute, but always acting on entire terms. 
\begin{screen}{1,2}
A + B D G + C D A;
@replace_match!(%)( D Q?? -> 1);
A + 1;
\end{screen}
Note the difference with \subscommand{substitute},
\begin{screen}{1,2}
A + B D G + C D A;
@substitute!(%)( D Q?? -> 1);
A + G + A;
\end{screen}
See \subscommand{take\_match} for further details on how to use this
``select/modify/replace'' mechanism.


\cdbseealgo{take_match}
\cdbseealgo{substitute}


\cdbalgorithm{take\_match}{}

Select a subset of terms in a sum or list which match the given
pattern. 
\begin{screen}{1,2}
A + B D G + C D A;
@take_match(%)( D Q?? );
B D G + C D A;
\end{screen}
In particular, note that the
{\tt Q??} is necessary to ensure that the pattern matches a product of
{\tt D} with something else. However, the algorithm has selected the
entire term, not just the part matched by the pattern; compare the
similar
\begin{screen}{1,2}
A + B D G + C D A;
@substitute!(%)( D Q?? -> 1);
A + G + A;
\end{screen}
in which the replacement is done on the pattern, not on the entire
term which contains the pattern.

This algorithm is particularly useful in combination with a copy
operation on the expression. It allows one to take out certain terms
from an expression, do manipulations on it, and then substitute it
back using \subscommand{replace\_match}.

The following example shows how this works by taking out the term
which contains a $\chi$ factor, doing a transformation on the $A_{m
n}$ tensor in that term, and then substituting back
using \subscommand{replace\_match}.
\begin{screen}{1,2,4,5}
expr:= A_{m n} \chi B^{m}_{p} + \psi A_{n p};
@take_match[@(%)]( \chi Q?? );
A_{m n} \chi B^{m}_{p};
@substitute!(%)( A_{m n} -> C_{m n} );
@replace_match!(expr)( \chi Q?? -> @(%) );
C_{m n} \chi B^{m}_{p} + \psi A_{n p};
\end{screen}
The \subscommand{replace\_match} pattern matching rules are identical
to those in \subscommand{take\_match}: a match always matches an entire
term, not just a factor of it.

\cdbseealgo{replace_match}
\cdbseealgo{substitute}


\input{zoom}

\printbibliography

\end{document}
