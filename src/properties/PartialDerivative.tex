\cdbproperty{PartialDerivative}{}

Makes an object a partial derivative, i.e.~a derivative which
commutes. The object on which it acts has to be a non-sub/superscript
child, while all the sub- or superscript child nodes are interpreted
to be the variables with respect to which the derivative is taken.
\begin{screen}{1,2,3,4}
\partial{#}::PartialDerivative.
A_{\mu}::Depends(\partial).
\partial_{\nu}{A_{\mu} B_{\rho}};
@prodrule!(%);
\partial_{\nu}{A_\mu} B_{\rho}
\end{screen}
Note that derivative objects do not necessarily need to have a sub- or
superscript child, they can be abstract derivatives as in
\begin{screen}{1,2,3}
D(d?)::PartialDerivative.
D(c d e);
@prodrule!(%);
D(c) d e + c D(d) e + c d D(e);
\end{screen}
If you want to write a derivative with respect to a coordinate (instead
of with respect to an index, as in the first example above), refer to
the \subsprop{Coordinate} property.

\cdbseeprop{Coordinate}
