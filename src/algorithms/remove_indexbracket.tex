\cdbalgorithm{remove\_indexbracket}{}

When a single symbol-with-indices is entered as
\begin{screen}{1}
(A)_{\mu\nu};
\end{screen}
it gets wrapped in an \texcommand{indexbracket} node. In order to
remove this node again, and produce
\begin{screen}{1}
A_{\mu\nu}
\end{screen}
again, use \subscommand{remove\_indexbracket}. 
\begin{screen}{1,2}
(A)_{\mu\nu};
@remove_indexbracket!(%);
A_{\mu\nu};
\end{screen}

This algorithm is also useful after the \subscommand{distribute}
command has
been called on an
\texcommand{indexbracket} node which contained a sum; the
\subscommand{remove\_indexbracket} will then remove the superfluous
brackets around the single symbols.

\cdbseealgo{distribute}
